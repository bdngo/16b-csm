\documentclass[aspectratio=169]{beamer}
\usepackage{amsmath, amssymb, amsfonts, amsthm}
\usepackage{cancel}
\usepackage[output-complex-root=j]{siunitx}
\usepackage[american, nooldvoltagedirection]{circuitikz}
\usepackage{bm}
\usepackage{lmodern}
\usepackage{listings}
\usepackage{graphicx}
\usepackage{hyperref}

\usetheme{Berkeley}
\usefonttheme[onlymath]{serif}
\AtBeginSection[]{
    \begin{frame}
    \vfill
    \centering
    \begin{beamercolorbox}[sep=8pt,center,shadow=false,rounded=false]{title}
    \usebeamerfont{title}\insertsectionhead\par
    \end{beamercolorbox}
    \vfill
    \end{frame}
}

\newcommand{\N}{\mathbb{N}}
\newcommand{\Z}{\mathbb{Z}}
\newcommand{\Q}{\mathbb{Q}}
\newcommand{\R}{\mathbb{R}}
\newcommand{\C}{\mathbb{C}}
\newcommand{\tpose}[1]{\left[#1\right]^{\! \top} \!\!}

\title{EECS 16B CSM}
\author{Bryan Ngo}
\date{2021-10-25}
\institute{UC Berkeley}

\begin{document}

\begin{frame}
    \maketitle
\end{frame}

\begin{frame}
    \frametitle{Logistics}

    \begin{itemize}
        \item CSM feedback form hopefully filled out
    \end{itemize}
\end{frame}

\begin{frame}
    \frametitle{Controllability}

    \begin{align}
        \bm{x}[t + 1] &= \bm{Ax}[t] + \bm{Bu}[t] \\
        \bm{x}[1] &= \bm{Ax}[0] + \bm{Bu}[0] \\
        \bm{x}[2] &= \bm{A}^2 \bm{x}[0] + \bm{ABu}[0] + \bm{Bu}[1] \\
        \bm{x}[t] &= \bm{A}^t \bm{x}[0] + \sum_{i = 0}^{t - 1} \bm{A}^{t - i} \bm{Bu}[i] \\
        \bm{x}[t] - \bm{A}^t \bm{x}[0] &=
        \begin{bmatrix}
            \bm{B} & \bm{AB} & \cdots & \bm{A}^{t - 1} \bm{B}
        \end{bmatrix}
        \begin{bmatrix}
            \bm{u}[t - 1] \\
            \bm{u}[t - 2] \\ 
            \vdots \\
            \bm{u}[0]
        \end{bmatrix} \\
        \Rightarrow \operatorname{span}\left\{
        \begin{bmatrix}
            \bm{B} & \bm{AB} & \cdots & \bm{A}^{t - 1} \bm{B}
        \end{bmatrix}\right\} &= \R^n
    \end{align}
\end{frame}

\begin{frame}
    \frametitle{Gram-Schmidt}

    \begin{align}
        \bm{p}_i &= \bm{v}_i - \sum_{j \neq i} (\tpose{\bm{v}_i} \bm{w}_j) \bm{w}_j \\
        \bm{w}_i &= \frac{\bm{p}_i}{\|\bm{p}_i\|}
    \end{align}

    \begin{itemize}
        \item Turn basis vector set \(V\) into orthonormal basis \(W\)
        \item Steps
        \begin{enumerate}
            \item Subtract all \(\operatorname{proj}_{\bm{v}_i}(\bm{v}_n)\) for \(i \neq n\)
            \item Normalize result
            \item repeat for all vectors
        \end{enumerate}
        \item systematically removing the parallel component of every other vector
        \item \href{https://upload.wikimedia.org/wikipedia/commons/e/ee/Gram-Schmidt_orthonormalization_process.gif}{cool GIF of GS}
    \end{itemize}
\end{frame}

\end{document}

\documentclass[aspectratio=169]{beamer}
\usepackage{amsmath, amssymb, amsfonts}
\usepackage[american, nooldvoltagedirection]{circuitikz}
\usepackage{hyperref}

\usetheme{Berkeley}

\usefonttheme[onlymath]{serif}

\title{EECS 16B CSM Presentation}
\author{Bryan Ngo}
\date{2020-08-24}

\begin{document}

\begin{frame}
    \maketitle
\end{frame}

\begin{frame}{Phasors}
\begin{itemize}
    \item Encodes information about any sinusoid: voltage, current, etc.
    \item If frequency is constant, then uniquely identifies
\end{itemize}
\begin{equation}
    A \cos(\omega t + \phi) = \Re\{A e^{j (\omega t + \phi)}\} = \Re\{\underbrace{A e^{j \phi}}_{\text{phasor}} e^{j \omega t}\}
\end{equation}
\end{frame}

\begin{frame}{Properties}
    Given \(x_1(t) = \Re\{A_1 e^{j \omega t}\}, x_2(t) = \Re\{A_2 e^{j \omega t}\}\) with phasors \(A_{1, 2}\),
    \begin{itemize}
        \item Uniqueness: \(x_1(t) = x_2(t) \implies A_1 = A_2\)
        \item Linearity: \(a_1 x_1(t) + a_2 x_2(t) \implies a_1 A_1 + a_2 A_2\) for \(a_{1, 2} \in \mathbb{R}\)
        \item Differentiation: \(x(t) \Leftrightarrow A \implies \frac{d}{dt} x(t) = \frac{d}{dt} \Re\{A e^{j \omega t}\} = \Re\{j \omega A e^{j \omega t}\} \Leftrightarrow j \omega A\)
    \end{itemize}
\end{frame}

\begin{frame}{Circuits \& Phasors}{KVL}
    \begin{center}
    \begin{circuitikz}\draw
        (0, 0) to[generic, v^=~] (0, 2) to[generic, v^=~] (2, 2) to[generic, v^=~] (2, 0) to[generic, v^=~] (0, 0);
        \draw[thin, <-, ] (1,1) ++(-60:0.5) arc (-60:170:0.5);
    \end{circuitikz}
    \end{center}
    \begin{equation}
        \sum_i \overline{V}_i = 0
    \end{equation}
\end{frame}

\begin{frame}{Circuits \& Phasors (cont.)}{KCL}
    \begin{center}
    \begin{circuitikz}\draw
        (-2, 2) to[generic, i=~] (0, 2) to[generic, i>^=~] (2, 2)
        (0, 0) to[generic, i=~] (0, 2)
    ;\end{circuitikz}
    \end{center}
    \begin{equation}
        \sum \overline{I}_{out} = \sum \overline{I}_{in}
    \end{equation}
\end{frame}

\begin{frame}{Circuits \& Phasors (cont.)}{Ohm's Law}
    \begin{center}
    \begin{circuitikz}\draw
        (0, 0) to[generic=\(Z\), v=\(V\), i>^=\(I\)] (4, 0)
    ;\end{circuitikz}
    \end{center}
    \begin{equation}
        \overline{V} = \overline{I} \underbrace{Z}_{\text{impedance}}
    \end{equation}
\end{frame}

\begin{frame}{Passive Elements \& Phasors}
    \scalebox{0.75}{
        \begin{circuitikz}\draw
            (0, 2) to[R=\(R\), v=\(V\), i>^=\(I\)] (0, 0)
        ;\end{circuitikz}
    }
    \begin{equation}
        \overline{V} = \overline{I} R
    \end{equation}
    \scalebox{0.75}{
        \begin{circuitikz}[scale=0.8]\draw
            (0, 2) to[L=\(L\), v=\(V\), i>^=\(I\)] (0, 0)
        ;\end{circuitikz}
    }
    \begin{equation}
        \overline{V} = L \frac{d}{dt} \overline{I} = j \omega L \overline{I}
    \end{equation}
    \scalebox{0.75}{
        \begin{circuitikz}[scale=0.8]\draw
            (0, 2) to[C=\(C\), v=\(V\), i>^=\(I\)] (0, 0)
        ;\end{circuitikz}
    }
    \begin{equation}
        \overline{I} = C \frac{d}{dt} \overline{V} = j \omega C \overline{V} \implies \overline{V} = \frac{1}{j \omega C} \overline{I}
    \end{equation}
\end{frame}

\begin{frame}{Demo}
    \Huge\url{http://tinyurl.com/y5qfnqtk}
\end{frame}

\begin{frame}{Low Pass Filter}
    \begin{align}
        \overline{V}_{out} &= \frac{\frac{1}{j \omega C}}{R + \frac{1}{j \omega C}} \overline{V}_{in} \\
        \overline{V}_{out} &= \frac{1}{1 + j \omega RC} \overline{V}_{in} \\
        \Rightarrow H(j \omega) &= \frac{\overline{V}_{out}}{\overline{V}_{in}} = \frac{1}{1 + j \omega RC}
    \end{align}
\end{frame}

\end{document}
